\documentclass[Berlin]{beamer}
\usepackage{listings}



\begin{document}

  \begin{frame}
    \frametitle{Buenas Practicas}
      Charla basada en el paper: 
      Principios que pueden ahorrarnos mucho tiempo
      Requiere esfuerzo cambiar las constumbres, pero vale la pena!

  \end{frame}


  \begin{frame}
    \frametitle{Usar control de Versiones}
    \framesubtitle{Vamos a usar GIT}
    crear un repositorio
    bajar material
    subir los trabajos prácticos

  \end{frame}

  \begin{frame}
    \frametitle{GIT}
    \framesubtitle{Sitema de control de versiones}
    ¿Para que? 
    Que pasa cuando mucha gente toca el mismo código/trabajo etc = LIO.
    más información:    http://es.wikipedia.org/wiki/Git
    
    Se puede usar con cualquier tipo de documentos, 
    pero se aprobecha su potencial en archivos de texto (de matlab por ejemplo)

  \end{frame}

  \begin{frame}
    \frametitle{GIT}
    \framesubtitle{Trabajando}
  
      Vamos a utilizar este tutorial:    %http://bit.ly/Tm0HVD

      crear un  directorio
      
      ir a una terminal 
%      \begin{lstlisting}[language=Bash]
%        git init 
%      \end{lstlisting}
%      

  \end{frame}


\end{document}
